\subsection*{Technological Complexity}
\phantomsection
\addcontentsline{toc}{subsection}{Technological Complexity}
\label{subsection:technologicalcomplexity}
The binary matrix \(M_{CT}\) forms the two metrics which characterize each node in the bipartite network by the sum of connections, known as degree centrality. The diversity of corporations \(K_{C,0}\) quantifies the number of technologies in which a corporation is specialized, and indicated by $M_{CT}$. 
Similarly, the ubiquity of technologies \(K_{T,0}\) represents the number of corporations that exhibit a technological advantage in a given field. These quantities are computed as

\begin{equation}\label{eq3}
\begin{aligned}
K_{C,0} = \sum_{T} M_{CT},  
K_{T,0} = \sum_{C} M_{CT}~.
\end{aligned}
\end{equation}

Higher diversity values indicate corporations with broad technological capabilities, while lower ubiquity values signify that the technologies are rare. 
Through calculating these two characteristics of adjacent nodes iteratively, the diversity and ubiquity of each node is updated as

\begin{equation}\label{eq4}
K_{C,N} = \frac{1}{K_{C,0}} \sum_{T} M_{CT}\, K_{T,N-1},
\end{equation}

\begin{equation} \label{eq5}
K_{T,N} = \frac{1}{K_{T,0}} \sum_{C} M_{CT}\, K_{C,N-1}.
\end{equation}

Here, the metric \(K_{T,1}\) represents the average nearest neighbor degree for technology nodes and it reflects the average diversity of corporations connected to a given technology.
Substituting Eq. \ref{eq4} into Eq. \ref{eq5} yields

\begin{equation} \label{eq6}
K_{T,N} = \sum_{T'} \widetilde{M}_{TT'} K_{T',N-2}
\end{equation}

where \(T'\) denotes a technological field, and $\widetilde{M}_{TT'}$ is defined as

\begin{equation} \label{eq7}
\widetilde{M}_{TT'} = \sum_{C} \frac{M_{CT}\, M_{CT'}}{K_{C,0}\, K_{T,0}}~.
\end{equation}

Eq. \ref{eq6} is satisfied when  \(K_{T,N} = K_{T,N-2} = 1\), which corresponds to the eigenvector associated with the largest eigenvalue of the stochastic matrix \(\widetilde{M}_{T,T'}\). 
Since this eigenvector is a vector of ones and not informative, the eigenvector corresponding to the second largest eigenvalue of \(\widetilde{M}_{TT'}\) captures the greatest variance among technology fields \cite{Hidalgo2021, Mealy2019}.
Thus, TCI is given by
\begin{equation} \label{eq8}
TCI = \frac{\widetilde{T} - \langle \widetilde{T} \rangle}{\text{stdev}(\widetilde{T})},
\end{equation}

where \(\widetilde{T}\) represents the eigenvector corresponding to the second largest eigenvalue of \(\widetilde{M}_{TT'}\), and \(\langle \widetilde{T} \rangle\) is their mean and \(\text{stdev}(\widetilde{T})\) is their standard deviation.
